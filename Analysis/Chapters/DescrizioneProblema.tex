\chapter{Descrizione del Problema}
L'algoritmo analizzato si occupa di effettuare un pattern matching. Il problema può essere formalizzato nel seguente modo: definito un insieme $\Sigma$ di simboli, detto {\itshape alfabeto}, si vuole comprendere se una sequenza di elementi dell'alfabeto $$\{x_0\:x_1\: ...\:x_{n-1} \;\; | \;\; x_i \in \Sigma\;\; 0\le i \le n-1\}$$ detta {\itshape pattern}, sia contenuta o meno all'interno di una sequenza di dimensioni maggiori.\\ Nel nostro caso l'algoritmo utilizza in input una stringa alfanumerica e un testo in cui effettuare la ricerca e restituisce l'indice del carattere iniziale della prima occorrenza del pattern alfanumerico.\\ 
Nel corso degli anni sono stati studiati ed implementati diversi algoritmi che si occupano di risolvere tale problema poiché quest'ultimo si presenta in campi di ricerca di vario genere. Per citare solo un esempio, nella bioinformatica l'algoritmo viene utilizzato nella ricerca di una determinata sequenza di basi all'interno del DNA.\\
L'algoritmo sequenziale brute force per la ricerca di stringhe scandisce il testo, utilizzando numerose interazioni e ripetendo confronti già effettuati. Ad ogni iterazione vengono confrontati in maniera sequenziale gli elementi del testo e quelli della stringa per vedere se è presente un'occorrenza. Al primo confronto che restituisce esito negativo si passa all'iterazione successiva in cui i confronti vengono ripetuti dall'inizio del pattern.\\ Quest'algoritmo è poco efficiente, in quanto non tiene conto, in nessun modo, delle informazioni acquisite nelle iterazioni precedenti. \\
Un algoritmo più complesso ed efficiente è, invece, quello di Knuth Morris Pratt (KMP). Questo è stato sviluppato da Knuth e Pratt e indipendentemente da Morris nel 1975. In questo caso, l'algoritmo ha prestazioni migliori in quanto riduce il numero di confronti necessari alla ricerca. Si fa uso di una tabella che viene calcolata preliminarmente ed in essa vengono salvate informazioni riguardanti la posizione, all'interno del pattern, da cui ricominciare il confronto. \\
In questo modo, non tutti gli elementi del pattern precedentemente trovati nel testo devono essere riesaminati. Sfruttando tutte le informazioni acquisite precedentemente, si riesce a diminuire il numero di confronti necessari nell'approccio classico.\\
L'algoritmo sviluppato all'interno della trattazione è una particolare implementazione parallela dell'algoritmo KMP che permette di lavorare in contemporanea su più sezioni del testo iniziale. \\ Il protocollo di progammazione multiprocessore che è stato utilizzato da questa variante dell'algoritmo di Knuth, Morris e Pratt,  è basato sullo scambio di messaggi e viene chiamato Message Passing Interface (MPI).\\
In seguito verranno analizzate le prestazioni e le misurazioni temporali effettuate al variare del numero di processori.\\